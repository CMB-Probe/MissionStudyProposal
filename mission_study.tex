
\subsection{Mission Study and Management Plan  }
\label{sec:management}

\vspace{-0.05in}


There are compelling science targets for an imager-based and for 
a spectrometer-based CMB Probe. Our study will investigate the science and technical trade-offs 
for both of these options, and for a combined mission. 

The study itself is open for the entire CMB community and already includes more than 40 scientists representing 
hundreds of years of experience with CMB theory, data analysis, and measurements on all platforms including satellite missions
that have already flown (WMAP, and \planck ) and the two proposed (LiteBIRD and CORE). 
Figure~\ref{fig:management} shows the management structure of the collaboration. The PI Hanany, who has more 
than 20 years of CMB ballooning experience, co-led MAXIMA and Archeops, was the PI of MAXIPOL and EBEX, and 
is a member of the CORE steering group, will have ultimate responsibility for the study. He 
is advised by a Steering Committee -- Bennett (Johns Hopkins), Dodelson (Chicago), and Page (Princeton) -- 
and assisted by a business office at the University 
of Minnesota.  An Executive Committee (EC) is in charge of the daily operation of the collaboration. The PI and each member of the 
EC also have responsibility for a particular study working group (WG), as outlined in the Figure. Significant overlap and feedback is 
expected among the WGs. 

%and as outlined in more detail below? 
%NASA/JPL, lead by CoI Lawrence (Chief Scientist 
%at JPL and US \planck\ team lead), will assist with mission design. The PI will lead the study of the Imager and will interface with 
%the JPL team. Al Kogut (PI of the Explorer-proposed PIXIE) will lead the spectrometer study. Knox and Flauger will lead the theory and 
%foregrounds aspects of the study; McMahon and Lee (PI of the US LiteBIRD team) will lead the study of relevant technologies.  

\begin{figure}[ht!]
\hspace{-0.1in}
\parbox{3.5in}{\centerline {
\includegraphics[width=3.5in]{../OrgChart_Names} } }
\hspace{0.05in}
%\end{center}
\parbox{2.5in}{
\caption{ \small \setlength{\baselineskip}{0.95\baselineskip}
Management structure of the CMB Probe. An Executive Committee led by the PI manages the work of the entire team. 
Membership in the team is open to all members of the CMB community. Working groups, led by members of the executive 
committee, investigate and develop aspects of the mission. 
\label{fig:management} } }
\vspace{-0.1in}
\end{figure}


The study will be carried out through intra- and inter-WG teleconferences; mission design teleconference with JPL engineers; 
mission design meetings at JPL; and a community workshop that is described in more detail below under the `Space / Sub-Orbital 
Synergy' WG. We now describe the planned work for each of the WG. 

$\bullet$ {\bf Theory (Knox)} \hspace{0.1in} This WG will survey, summarize, and prioritize the set of 
science goals for the Probe.  Given input on target frequency bands and instrument noise levels the group will 
generate forecasts for the impact of the Probe's products and their ultimate 
significance for physics and astrophysics.

$\bullet$ {\bf Mission (Hanany) and connection with JPL (Lawrence)} \hspace{0.1in} The Mission WG is responsible 
for defining the overall mission 
architecture including telescope implementation, cooling, telemetry, mass, power, and cost. The WG will work closely 
with the JPL lead scientist (Lawrence) and JPL mission engineers. 

$\bullet$ {\bf Imager (Hanany) and Spectrometer (Kogut)} \hspace{0.1in} The imager and spectrometer WGs will 
translate the science goals to 
mission requirements and to a set of optional designs. The designs will include telescopes of various configurations, 
focal planes with several candidate detector technologies and readout schemes. These groups will similarly consider 
the options for spectrometers.  Both groups will work closely 
with the mission WG and with the JPL team to assess the relative merits 
of the optional designs, which will include an imager-only and spectrometer-only options, as well as a combined
instrument.  

$\bullet$ {\bf Space / Sub-Orbital Synergy (Jones, Devlin)} \hspace{0.1in} By the time the \ac{CMB} probe is likely to fly,
significant advances are expected to be made on the ground. This is true regardless of the state of the proposed CMB-S4
effort, and even more so should funding for S4 becomes available soon. This working group will assess and recommend the 
most appropriate design parameters such that the data sets from the Probe and sub-orbital measurements complement each other. 
Pertinent questions include: to what extent should the aperture size of the Imaging Probe rely on delensing capabilities provided by 
high resolution measurements from the ground? What is the optimal resolution of a space based mission from the point 
of view of providing foreground subtraction capabilities to sub-orbital missions? What is an optimal overlap in $\ell$-space
coverage? Does the design of a spectrometer depend on the specifics of data available from sub-orbital measurements? 

We are planning a community workshop to address these question, including forming a community consensus on the 
question of the need for a space mission if CMB-S4 is funded. 

$\bullet$ {\bf Complementarity with Other Data Sets (Pryke)} \hspace{0.1in}
The full sky nature, the broad frequency coverage, and the high sensitivity of the CMB-Probe will generate legacy 
data set surpassing that of \planck 's. This working group will survey the possible cross-correlations with astrophysical 
data available at other wavelengths. It will assess whether such cross-correlations can benefit by preferring 
some mission parameter values over others. Examples include adjusting the resolution, and frequency coverage. 
The group will also survey possible sources of systematic uncertainties and how these can be addressed during mission 
design, implementation, and data analysis. 

$\bullet$ {\bf Systematics (Crill)} \hspace{0.1in}

$\bullet$ {\bf Foregrounds (Flauger) } \hspace{0.1in}

One of the key ingredients in the design of a CMB experiment is the frequency coverage required to achieve the science goals. Consequently, optimizing frequency coverage in light of the new information from $Planck$ and its limitations will be one of key task of the study proposed here. 

%OVERVIEW of PLAN
To achieve these goals we plan a careful investigation of the effect on the measurements of r and $\tau$ of the presence of foreground residuals in the CMB maps (after foreground separation and/or cleaning) including, for example, the properties of the polarized thermal dust emission, specifically the spatial variation of its spectral index (hinted at by the observed decorrelation of the dust between 217 GHz and 353GHz Planck channels \ref{Planck2015-X;Planck2015-L;Planck2015-XXIX;Boulanger2016}), the breakdown of the modified black body spectrum model, and decorrelation between frequencies. 

These aspects will be explored with the help of physically motivated models of the foregrounds (\ref{Bruce+Fraisse2009,Hensley et al in preparation}) informed by existing data along with simulations based on these models. To incorporate instrument systematics due to beams, bandpass mismatches, correlated noise, etc., time domain based simulations will be required.

%INSTRUMENTAL PARAMETERS:
To devise the optimal instrumental/observational concept/design we will probe several configurations by varying critical instrumental parameters such as: the frequency coverage, the number of frequency channels, inter-band separation (limited by the typical instrumental bandwidths), the noise or sensitivity of each frequency channel, and angular resolution. 

%FG MODELLING:
Given that the B-mode signal is very weak, and much fainter than galactic foreground emission, even slight inacuracies in the characterization of the foregrounds will potentially impact the detectability of the B-mode signal. Therefore it is crucial to consider a large frequency coverage to properly gather the nature of the foregrounds as well as 'realistic' models that capture intrinsic complexities of these diffuse polarized foregrounds.
Therefore we will consider physically motivated dust models (beyond the modified blackbody spectrum approximation) currently in development. 
For these models the absorption cross-section in Intensity and Polarization are estimated self-consistently for grains of different sizes, shapes, and compositions as a function of frequency. These physical dust models will be used along with models implemented in the Planck Sky Model, PSM \ref{}, and/or  Python Sky model, PySm \ref{}, 


%TECHNIQUES:
To forecast the performance of a given instrumental design we will resort, for a quick diagnosis, to traditional techniques such as Fisher codes, both spectra and map based, that account for the presence of foregrounds assuming some best fit model of both the CMB and Foregrounds (akin to those used for the CMB-S4 science book \ref{}, eg CMB4cast (http://portal.nersc.gov/project/mp107/index.html)). 

%SIMULATIONS
For a more in depth analysis, that can also probe biases in the parameters, we will simulate maps of the CMB and Foreground emission at each frequency with CAMB and PSM (or PYSM) packages respectively and HEALpix modules \ref{}. Noise simulations will then be added to the signal maps.
While white noise or anisotropic noise are straightforwardly simulated directly on map domain, 1/f or correlated noise requires simulating time ordered data according to a noise prescription or generator (eg akin to LevelS \ref{} used in Planck). The convolution of the simulated maps with the Gaussian beams can be performed straightforwardly with HEALpix modules, while the convolution with more realistic beams is harder but can be performed with approaches such as FEBeCOP \ref{}, developed for Planck data analysis. To apply the latter an optical beam and scanning strategy needs to be specified and adopted by the software.

The next step is to clean the frequency maps from foregrounds and generate the 'clean' CMB map, using techniques such as Commander, SMICA, SEVEM and possibly NILC \ref{Planck2015-IX}. This is followed by estimation of auto and cross-correlation angular power spectra of the maps using say Master based \ref{}  or XFaster \ref{} based techniques and propagated to parameter estimation using CosmoMC or Multinest sampling.
Along with this standard procedure we will also apply a novel technique based on direct Bayesian MCMC inference of cosmological parameters in the presence of foregrounds, without resorting to Likelihood approximations (an extension of the method presented in  \ref{Jewelletal2016}). 
The latter will be integrated with Commander, allowing to bypass the angular power spectrum estimation as it samples Cosmological and Foregrounds parameters directly from the simulated maps. As in this approach the foreground parameter fits (including the spectral index) is performed pixel by pixel, the spatial variation of the spectral index is naturally accounted for. 

%As mentioned earlier 1/f or correlated noise requires simulating time ordered data. To analyse the resulting maps another ingredient is needed, the pixel-to-pixel noise correlation matrix. For a large number of pixels the estimation of this matrix is computationally intensive. As 1/f and correlated noise leaves mostly in the large angular scales and we resort to studying these effects on low resolution maps (reducing computational costs), hybridized with higher resolution maps for the white noise component. 

It should be stressed that to fully assess the performance of a given instrument design, in view of both foreground residuals and the presence of systematics, realistic simulations are essential. These include time domain based simulations which can be generated using HPC4CMB ((https://github.com/hpc4cmb)) based on TOAST 
%(data framework, including on-the-fly simulation capabilities with full detector beam, bandpass and noise properties)), 
and a destriper based map-making algorithm such as libMADAM. 


%FOM
Finally to quantify performance we will need to define a figure of merit, FOM. Examples of possible FOM, are: the effective noise in the I,Q,U maps after foreground cleaning (effectively quantifying the noise degradation due to the presence of residual foregrounds); uncertainty and biases in the parameters; the evidence of the best fit model for each instrumental design (used as a qualifier of the instrumental design itself), etc.


\comred{GR:This is a first version of the plan - still needs editing} 
 




$\bullet$ {\bf Technology (Lee, McMahon)} \hspace{0.1in}


Sub-orbital measurements are complementary to those made aboard a space probe. By the time the \ac{CMB} probe is likely to fly significant advances are expected to be made on the ground. 

