
\subsection{Science Objectives}
\label{sec:science}

\vspace{-0.05in}

\comred{5 pages for all science goals including the (temporary) two sections below.}

\subsubsection{The Inflationary Gravitional Wave Background}

\vspace{-0.05in}

\comred{The verbiage below is taken from another proposal. Here we need to explain what are the science objectives 
of the CMBProbe, how the science objectives relate to the current state of knowledge, and to NASA's goals}
% Lloyd, Sarah, Dan, Rafael

The paradigm of inflation~\cite{guth81,linde82,albrecht82,sato81,kolb94}
%, in which the Universe underwent exponential expansion within the first $\sim$$10^{-35}$~sec, 
makes several predictions that are consistent with all current astrophysical 
measurements~\cite{spergel06,Tegmark:2006az,planck2015parameters,planck2015inflation}. 
A robust prediction of inflation is the existence of a stochastic background of gravitational radiation 
with an amplitude depending on the mechanism driving the accelerated 
expansion~\cite{starobinsky82,starobinsky83a,rubakov82,grishchuk75,abbott84a}.
In most scenarios, this `inflationary gravitional wave background' (\igb) is predicted
to have a spatial power spectrum whose amplitude is proportional to the energy
scale of inflation $V^{1/4}$ via
$V^{1/4} = 3.7 \times 10^{16} \ r^{1/4}\,\, {\rm GeV},$
where $V$ is the inflaton potential and $r$ is the ratio of the temperature
quadrupoles produced by gravitional waves and by density perturbations.  
There are theoretical reasons $V^{1/4}$ may be close to the Grand
Unification scale of $10^{16}$~GeV, suggesting detectable $r$ values between 
$\sim$0.001 and $\sim$0.1. In addition to determining the energy scale of inflation, measurements 
of the \igb\ probe the scalar field potential at or above the Planck scale, which is particularly relevant for inflation models motivated 
by string theory~\cite{SnowmassInflationTheory}. Measurements of the \igb\ thus probe fundamental physics at the 
highest possible energy scales. 
\begin{figure}[htbp!]
\hspace{0.in}
\parbox{4.2in}{ \centerline {
\includegraphics[width=2.0in] {Figures/sunny_skies.jpg}  
\hspace{0.1in}
\includegraphics[width=2.0in] {Figures/sunny_skies2.jpg}  }  }
\hspace{0.1in}
\parbox{2.in}{
\caption{ \small \setlength{\baselineskip}{0.90\baselineskip}
       Sample Figure of Sunny Skies
\label{fig:sunny_skies} } }   
\vspace{-0.05in}
\end{figure}

The most promising way to search for the \igb\ is through its signature on the CMB polarization~\cite{kamionkowski97b,seljak97}.  
Primordial energy density perturbations produce only a curl-free, or `E-mode', pattern of polarization.
Gravitional waves also produce a curl, or `B-mode', pattern of polarization that density perturbations cannot
produce~\cite{kamionkowski97a,zaldarriaga97}.  The amplitude of the B mode is related to the energy scale
of inflation by $V^{1/4}=2\times10^{16} \ ( B_{peak} / 0.1\,\mu{\rm
K})^{1/2} \,{\rm GeV},$ where $B_{peak}$ is the amplitude of the power spectrum of the B mode in \microk\ at $\ell=80$;
see Fig.~\ref{fig:sunny_skies}. In its recent report New Worlds New Horizons (NWNH), the decadal survey 
committee strongly endorsed sub-orbital searches for the B-mode signal from 
inflation saying that ``The convincing detection of B-mode polarization in the CMB produced in the 
epoch of reionization would represent a watershed discovery.''~\cite{blandford2010}

B-mode signatures near the expected \igb\ peak at $\ell=80$ have recently been detected by BICEP2~\cite{bicep2Bmode}. 
However, the combination of Planck data with those from the BICEP2 and Keck Array collaborations have demonstrated 
that the B-mode signal measured is entirely consistent with contributions from polarized emission of Galactic dust and the 
signal from the gravitational lensing of CMB photons by the large scale structure of the Universe (see 
Section~\ref{sec:lensing})~\cite{bkp2015,planck2014-XXX,2016PhRvL.116c1302B}. 
These data give an upper limit of $r<0.09$ at 95\% confidence level.
Most importantly, the constraint is largely limited by Planck's noisy measurement of the dust properties in the 353~GHz band; 
a noiseless dust map could shrink the constraint by a factor of two~\cite{bkp2015}. 
Further progress --- detections or improved limits --- requires instruments 
with higher sensitivity at {\it both} the dust and CMB frequency bands so that this Galactic foreground can be properly identified 
and removed. 

\vspace{-0.15in}

\subsubsection{Neutrinos and Light Relics}

\vspace{-0.05in}

Much of the information about our thermal history and the particle content of the universe is encoded in the $T$ and $E$ power spectra.  
A high-precision measurement of these spectra over the full sky is expected to significantly improve our understanding of the post-inflationary 
universe.  This is particularly true in $E$-mode polarization where, to date, far fewer modes have be measured at the level of cosmic variance than in temperature.

The spectra at high-$\ell$ contain important information about the components of the thermal plasma and their interactions around the time of recombination.  One particular compelling target is the effective number of neutrino species, $\Neff$, which parameterizes the total amount of energy density in radiation at the time of recombination.  It is defined such that in the Standard model of particle physics with normal thermal evolution, $\Neff = 3.046$ due to the energy density in the three species of neutrinos.  $\Neff$ is also sensitive to any additional light relic particles as their gravitational influence is identical to the neutrinos.  In fact, if there was an additional light particle in thermal equilibrium with the Standard model particles at any point in our history, it will contribute a change to $\Neff$ of at least $\Delta \Neff \geq g \, 0.027$ where $g \leq 1$ is the number of degrees of freedom of the new particle.  This defines a compelling target of $\sigma(\Neff) < 0.027$ for future CMB observations.  New light particles are a common feature of many approaches to beyond the Standard model physics and can be directly tied to some of the most significant problems in the Standard model.  Either a limit or detection of $\Delta \Neff$ at this level would provide a powerful insight into the laws of nature and our thermal history. 

The presence of free-streaming radiation changes the detailed features of the $TT$, $TE$ and $EE$ spectra at all $\ell$.  In particular, it changes the locations of the acoustic peaks and alters the damping tail at high-$\ell$.  Similar changes to the spectra arise from many other compelling targets including the helium fraction $Y_p$ and more general dark sector physics.  For this reason, constraints on $\Neff$ a useful proxy for the information available in the high-$\ell$ power spectra.  

Preliminary forecasts for $\Neff$ are shown in the right hand panel of Figure~\ref{fig:Neff_future}.  A space-based mission reaching an effective temperature noise of 1-2 $\mu$K-arcmin over the full sky gives competitive constraining power when compared other proposals.  The two most important quantities for improving constraints on $\Neff$ and other high-$\ell$ targets are $f_{\rm sky}$ and the temperature noise.  The full-sky nature of the proposed mission would allow for cosmic variance limited $E$-modes over most of the sky and a large range of $\ell$.

The main downside of a space-based mission is that we cannot reach the resolutions available from the ground.  However, we see that at 5' resolution and 1 $\mu$K-arcminute noise the forecasts are less sensitive to the resolution then one might naively expected.  In particular we can reach $\sigma(\Neff) < 0.035$ for temperature noise from 1-2 $\mu$K-arcmin and $f_{\rm sky} =0.6-0.8$.  These forecasts are competitive with CMB Stage IV.  Specifically, the larger sky fraction and sensitivity available from space appears to compensate for the reduced resolution.  In fact, the full sky measurement would provide complimentary information that could be combined with ground based surveys to further improve over the limits available from either experiment.  This is particularly important for $\Neff$ which is tantalizingly close to the target of $\sigma(\Neff) =0.027$ and therefore even an apparently modest improvement could have a major scientific impact.  

The sum of neutrino masses, $\sum m_\nu$, is another theoretically compelling target that is accessible from Cosmology.  The most distinctive feature of $\sum m_\nu$ is that it suppresses the growth of structure on small scales.  This suppressed can be measure in the CMB through amplitude of the lensing power spectra compared to the primary CMB.  In principle, this relative difference can yield a measurement  of the minimum value of $\sum m_\nu =58$ meV at 4-5 $\sigma$ for a number of future cosmological surveys.  However, sensitivity to $\sum m_\nu$ is ultimately limited by our knowledge of the primordial amplitude of fluctuations $A_s$ which is strongly degenerate with the optical depth $\tau$. 

The current limit on $\tau$ from the Planck satellite of $\tau = 0.055 \pm 0.009$ ultimately limits $\sigma(\sum m_\nu) \gtrsim 25$ meV, as shown in the panel of Figure~\ref{fig:Neff_future}.  While the figure shows the sensitivity of a space-based CMB mission to $\sum m_\nu$, this lower limit is common to any measurement that depends on the relative suppression.  Therefore, a cosmological detection of $\sum m_\nu = 58$ meV at 3-5 $\sigma$ depends crucially on an improvement measurement of $\tau$.  To date, the only proven method for such a measurement is from a space-based CMB observations.  The best constraints on $\tau$ come from $E$-modes with $\ell < 20$ which requires control over the largest angular scales.  


\begin{figure}[t!]
\begin{center}
\includegraphics[width=0.45\textwidth]{figs/Mnu_tauprior.pdf}
\includegraphics[width=0.45\textwidth]{figs/Neff.pdf}
\caption{ {\it Left:} Neutrino mass constraints as a function of the prior on $\tau$ for a 5' beam and sky fraction of $f_{\rm sky} = 0.7$.  The blue dashed line is the Planck blue book expectation and the white dashed line a cosmic variance limit measurement of $\tau$ form the CMB. {\it Right:} $\Neff$ Forecasts as a function temperature noise and sky fraction assuming 5' resolution.}
\label{fig:Neff_future}
\end{center}
\end{figure}



\vspace{-0.15in}
\subsubsection{CMB spectral distortion science}
\vspace{-0.05in}

In addition to the CMB temperature and polarization anisotropies targeted by CMB imagers, {\it unique} new information about early-universe physics can be gained by studying the energy spectrum of the CMB \citep{Sunyaev1970SPEC, Burigana1993, Hu1993, Chluba2011therm}. The measurements of COBE/FIRAS have shown that the average CMB spectrum is extremely close to that of a blackbody at a temperature $T_0=(2.726\pm 0.001)\,{\rm K}$ \citep{Mather1994, Fixsen1996}. However, several standard processes are expected to {\it distort} the CMB spectrum \citep[e.g.,][]{Chluba2016LCDM} at a level that is within reach of present-day technology \citep{Kogut2011PIXIE, PRISM2013WPII}. 
%
The classical distortion shapes are known as Compton-$y$ and chemical potential ($\mu$-type) distortions \citep{Zeldovich1969, Sunyaev1970mu} and are caused by energy exchange of CMB photons with free electrons. A $\mu$-distortion can only be produced in a hot and dense environment present at redshifts $z\gtrsim 5\times10^4$, while $y$-type distortions appear at lower redshifts. This makes $\mu$-distortions a unique messenger from the early Universe.

The largest guaranteed distortion is caused by the late-time energy release of forming structures and from reionization \citep{Sunyaev1972b, Hu1994pert, Oh2003, Cen1999, Refregier2000}, imprinting a $y$-type distortion with $y \simeq 2\times 10^{-6}$ \citep[e.g.,][]{Refregier2000, Hill2015}. This distortion is only one order of magnitude below the current limit from COBE/FIRAS and, even with most pessimistic assumptions about foregrounds, should be clearly seen with next-generation spectrometers, telling us about the total energy output of first stars, AGN and galaxy clusters. In particular, group-size clusters ($M\simeq 10^{13}\,M_{\odot}$) contribute significantly to the signal. These are still sufficiently hot (temperature $k T_{\rm e}\simeq 1\,{\rm keV}$) to create a visible relativistic temperature correction to this large $y$-distortion, which could be used to constrain cluster feedback models \citep{Hill2015}. These two inevitable signals probe the low-redshift Universe and provide clear targets for future spectral distortions measurements and their requirements in the presence of foregrounds.

Next generation CMB spectrometers are also expected to greatly improve the $\mu$-distortion limits of COBE/FIRAS \citep{Kogut2011PIXIE}. This will allow us to place stringent bounds on the presence of long-lived decaying particles \citep{Hu1993b, Chluba2013fore, Chluba2013PCA, Dimastrogiovanni2015} and other new physics \citep[e.g.,][]{Jedamzik2000, Tashiro2012, Dolgov2013, Tashiro2013, Caldwell2013, Yacine2015DM}, but a clear target is predicted by the dissipation of small-scale perturbation through Silk-damping \citep{Sunyaev1970diss, Daly1991, Hu1994, Chluba2012}. This process allows us to place stringent constraints on the amplitude of the small-scale curvature power spectrum, present at scales (wavelength $0.1 \,{\rm kpc} \lesssim \lambda \lesssim 1\, {\rm Mpc}$) and epochs ($10^4 \lesssim z\lesssim 10^6$) inaccessible through any other observation. This delivers a complementary test for the inflation paradigm \citep{Chluba2012inflaton, Dent2012, Chluba2013PCA, Clesse2014, Cabass2016}, with $\mu=(2.0\pm0.14)\times 10^{-8}$ expected in $\Lambda$CDM \citep{Chluba2016LCDM}. Precise measurements of signals at this level will be extremely challenging and requires unprecedented control of systematics and modeling of foregrounds. It would also bring us to the sensitivity level required to detect the cosmological recombination radiation \citep{Sunyaev2009, Chluba2016} imprinted by the recombination of hydrogen and helium at redshift $z\simeq 10^3-10^4$. Optimizing next-generation CMB spectrometers for these purposes requires extensive studies.

Aside from the average CMB distortion signals, the CMB spectrum can also vary across the sky. One source of anisotropic distortions is related to clusters of galaxies and has already been measured \citep{Planck2013SZ}. A combination of precise CMB imaging and spectroscopic measurements might allow observing the relativistic temperature correction  \citep{Sazonov1998, Itoh98, Challinor98} of individual SZ clusters. This could allow us to calibrate cluster scaling relations and learn about the dynamical state of the cluster atmosphere. Anisotropies in the $\mu$-distortion can be created through ultra-squeezed limit non-Gaussianity \citep{Pajer2012, Ganc2012} and could be used to probe scale-dependent non-Gaussianity \citep{Biagetti2013, Razi2015}. Finally, resonant scattering signals in the recombination \citep{Jose2005, Carlos2007Pol, Lewis2013} and post-recombination eras \citep{Kaustuv2004, Schleicher2008} can lead to spectral-spatial CMB signals that can be used to constrain the presence of metals in the dark ages and the physics of recombination. For all these applications, instrumental synergies between CMB imaging and spectroscopy need to be studied in detail. 


In summary, future studies of the CMB spectrum will open a new {\it unexplored} window to early phases of the Universe ($z > 10^3$), which cannot be probed in any other way. This will not only allow us to test the standard cosmological paradigm (e.g., inflation and reionization) but also opens up a huge discovery space to non-standard physics (e.g., decaying/annihilating particles). This immense potential and complementarity with CMB anisotropy studies makes CMB spectral distortions an important future target and identifying experimental routes towards extracting these tiny signals from the early Universe will be one main objective of the proposed mission study.
