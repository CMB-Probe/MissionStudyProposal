
\subsection{Current and Forthcoming Efforts and the CMB Probe}
\label{sec:spacemission}

\vspace{-0.05in}

The remarkable scientific yield resulting from past CMB observations
has motivated significant investments in current and future
experiments which are designed to realize the full potential of this
unique probe of fundamental physics.  These ground-based and
balloon-borne experiments are designed to exploit the comparative
advantages of the sub-orbital platforms over an orbital mission, while
providing the design heritage and experience necessary to maximize the
probability of success of an orbital mission.

For the ground-based efforts, these include combinations of {\it i)}
provision for large primary apertures and therefore high angular
resolution, {\it ii)} extremely long observation times, {\it iii)}
flexibility to rapidly deploy new technologies, and {\it iv)}
allowance for detector formats that are relatively unconstrained by
mass and power limitations.  In aggregate, these enable extremely low
noise measurements of small scale anisotropies with high redundancy
over large areas of the sky.

The balloon-borne missions {\it i)} extend the frequency reach of the
ground based telescopes, {\it ii)} enable high fidelity measurements
on larger angular scales than can be probed from the ground, and {\it
  iii)} grant inexpensive access to a space-like environment,
providing heritage for future space missions as well as experience in
dealing with the analysis of data that are representative of a space
mission.

In this way, the sub-orbital programs complement and multiply the
scientific return of the proposed orbital mission, while reinforcing
its technical preparedness.  A robust program of sub-orbital
experimentation has proven a vital component in the success of all
three generations of previous CMB orbital missions -- COBE, WMAP and
{\it Planck}. Building on this heritage, the current (Stage-3) and
planned (Stage-4) sub-orbital experiments are well poised to play a
similar role for the CMB Probe, which will provide definitive
measurements of the full sky from the largest angular scales to the
$5^\prime$ scale of the beam.

Ground based experiments are already approaching the level of
sensitivity of the CMB Probe over limited portions of the sky ($<1$\%
of the full sky), and a limited range in frequency.
%cite bicep/keck, spt
Balloon-borne experiments and ground based telescopes at
mid-latitudes are now beginning to extend these measurements to
significant fractions of the full sky. 
%cite ACT
Currently funded balloon borne
experiments will add to this sensitive data at frequencies above 200
GHz which will help characterize Galactic dust, while ground based
experiments in Chile will add critical information at lower
frequencies to constrain synchrotron emission.
%cite Spider Fraisse et al, Piper, Ebex, CLASS, ACT
The Stage-4 experiments
will extend these measurements to much larger areas, and to angular
scales smaller than $5^\prime$, with benefits to the CMB Probe as
described in Section \ref{sec:science}.

The CMB Probe will add to these complete sky coverage, fidelity to the
largest angular scales, comprehensive frequency coverage and exquisite
control of systematic effects.  As evidenced by the experience of
Bicep and {\it Planck}, the joint analyses are becoming increasingly
important to fully exploit the scientific value of CMB data sets.  As
part of this mission study, we will organize a workshop to organize
this effort, and investigate the ways in which the ground and
sub-orbital balloon programs can best complement the CMB probe.
