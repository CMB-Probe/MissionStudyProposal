
\subsection{The CMB Probe in Context}
\label{sec:spacemission}

\vspace{-0.05in}

\subsubsection{Current and Forthcoming Sub-Orbital Efforts}

The remarkable forthcoming scientific yield has motivated significant agency investments 
in current and future sub-orbital experiments which are designed to realize the full potential of this
unique probe of fundamental physics and astrophysics.    These experiments are designed 
to exploit the comparative advantages of the sub-orbital platforms, while providing the design heritage and 
experience necessary to maximize the probability of success of an orbital mission. 

For the ground-based efforts, these include combinations of {\it i)}
provision for large apertures and therefore high angular
resolution, {\it ii)} flexibility to rapidly deploy new technologies, and {\it iv)}
allowance for detector formats that are relatively unconstrained by
mass and power limitations.  To date, these have demonstrated low
noise measurements of small and intermediate angular scale $E$ and $B$ polarization 
structures over less than 2\% fractional areas of the sky. 


The balloon-borne missions {\it i)} extend the frequency reach of the ground based telescopes, 
{\it ii)} enable high fidelity measurements on larger angular scales than can be probed from the 
ground, and {\it iii)} grant access to an environment with similar requirements and constraints
as in orbit, providing heritage for future space missions as well as experience in dealing with the 
analysis of data that are representative of a space mission. In this way, the sub-orbital programs 
complement and multiply the scientific return of the proposed orbital mission, while reinforcing its 
technical preparedness.  

The 2010 Decadal Panel strongly recommended supporting sub-orbital efforts in preparation 
for a possible space mission to follow sub-orbital detections of inflationary gravity waves. 
As a result, the US has clear leadership in the field, both in terms of ground- and balloon-based 
experiments and results. 

This leadership will continue into the foreseeable future. In aggregate, funded, now-being-built 
`Stage 3' CMB experiments will deploy approximately 100,000 detectors on various sub-orbital 
experiments within the next 3-5 years. 
Ground-based experiments plan to extend measurements from few percent of the sky to 
few tens, although in a limited frequency range between 30 and 300~GHz. Balloon-borne 
payloads operating at even higher frequencies strive to cover even larger fractions.  

%As a result, noise levels will decrease, the foregrounds will progressively be better understood, 
%technologies will be tested, and we will have learned how to 

%Some funded ground-based efforts now being under construction or commissioning, are expected 
%to reach noise levels of ?? $\mu$ K arcmin over ??\% of the sky, in a limited frequency 
%range, between 90 and 280~GHz, and after ?? years of integration. 
%cite bicep/keck, spt
%Balloon-borne experiments plan to extend the measurement to much larger sky fractions, 
%up to 80\% in one case, and extend the frequency range to 600~GHz, 
%significant fractions of the full sky. 
%cite ACT
%Currently funded balloon borne
%experiments will add to this sensitive data at frequencies above 200
%GHz which will help characterize Galactic dust, while ground based
%experiments in Chile will add critical information at lower
%frequencies to constrain synchrotron emission.
%cite Spider Fraisse et al, Piper, Ebex, CLASS, ACT
%The Stage-4 experiments
%will extend these measurements to much larger areas, and to angular
%scales smaller than $5^\prime$, with benefits to the CMB Probe as
%described in Section \ref{sec:science}.

\vspace{-0.18in}

\subsubsection{Proposed Efforts: LiteBIRD, CORE, and CMB-S4} 

\vspace{-0.05in}

Japan, in collaboration with NASA, is now considering whether to proceed with LiteBIRD, a space mission 
designed to search for $B$ modes from inflation. The US Team has submitted its Phase A report to NASA; Phase A 
in Japan will conclude in about a year \comred{check}. LiteBIRD is a smaller, more focused
mission compared to the CMB Probe. It is an imager based on a 0.5~m aperture 
telescope. Therefore it has a resolution 4 times lower compared to the 2~m aperture of EPIC-IM. Its
reach in $\ell$ space is correspondingly 4 times lower making the science available at $\ell$'s above 
few hundred in both $E$ and $B$ modes unreachable. 
It has no spectroscopic capabilities and thus not sensitive to any of the spectral distortion science goals. 

For the Japanese space agency JAXA, LiteBIRD is meant to fit within the \$300M class of missions. 
Although there are uncertainties about comparing JAXA's cost calculations to NASA's, LiteBIRD's overall size 
and more limited science reach is commensurate with it being below, or just at the lower margin of the Probe's
cost window. 

A collaboration of scientists in Europe has just recently proposed CORE to ESA as part of the M5 round 
of space mission proposals. 
The team includes a number of US collaborators; the PI of this proposal is a member of 
CORE's Executive Board. CORE is a CMB polarization imager that is based on a 
1.2~m aperture telescope and thus intended to reach 2.5 times the resolution of LiteBIRD. ESA has 
capped the M5 proposals to EU550M, the equivalent of 
\$610M. Member countries are expected to contribute an additional $\sim$\$163M making the total 
cost close to \$773M. Selection of missions for Phase A studies is expected in fall 2017, and 
end of Phase A selection in fall 2019. 

The US CMB community has proposed, and the Particle Physics Project Prioritization Panel (P5) has recommended 
to the DOE, the establishment of a 4th generation CMB experiment called CMB-S4. This is an ambitious 
program to field approximately 5 times the number of detectors fielded by Stage 3 experiments. If and when funded, 
CMB-S4 will enable unprecedented sensitivity at frequency bands accessible from the ground, and 
with telescopes that enable high resolution. 

%A robust program of sub-orbital experimentation has proven a vital component in the success of all three generations of previous CMB orbital missions -- COBE, WMAP and {\it Planck}. Building on this heritage, the current (Stage-3) and planned (Stage-4) sub-orbital experiments are well poised to play a similar role for the CMB Probe, which will provide definitive measurements of the full sky from the largest angular scales to the $5^\prime$ scale of the beam.

\vspace{-0.18in}

\subsubsection{Why Study a CMB Probe?} 

\vspace{-0.05in}

Learning from the successes of COBE/FIRAS, COBE/DMR, WMAP, and \planck, a
CMB Probe is the single most suitable vehicle to deliver complete sky coverage 
and therefore information on the largest angular scales, 
comprehensive frequency coverage, and exquisite control of systematic effects. 
Some of the science goals described in Section~\ref{sec:science} 
are reachable only through mapping of the largest angular scales. No sub-orbital experiment 
has yet produced any polarization results on more than 2\% of the sky, let alone 
on scales requiring 70\% of the sky. The broad frequency coverage of the space 
mission is best suited to mitigate the foregrounds expected on a broad range of angular 
scales, including those important for removing the effects of B-modes from lensing. 
The mission will provide a single self-consistent and self-calibrated data set;  and it  
will provide legacy maps at many frequency bands that will become the basis for 
hundreds of new papers. 

If the Inflationary signal is detected by sub-orbital experiments
any time soon, a space mission to characterize the signal in full detail is equally compelling. 
The existence of ambitious sub-orbital programs is a complementary strength. How 
to make the best use of this complementarity is an explicit goal of our study; 
see Section{sec:management}.

\vspace{-0.18in}

\subsubsection{Does the CMB Probe Fit Within the Cost Window?} 

\vspace{-0.05in}

The total cost estimate for the EPIC-IM mission, as generated by JPL's Team X was \$920M in 2009~\cite{}. 
The mission had a 1.4 m effective entrance aperture. When the mission was assessed
by the 2020 Decadal Panel the independent cost estimate was \$1200M \comred{check}. The CORE mission, that had just been 
proposed to ESA and has an aperture of 1.2~m, was estimated by the proposing team to have a total cost \$773M. The cost
estimate for LiteBIRD, which has a 0.5~m aperture, is \$??. 

When NASA proposed to initiate studies for next-decade flagship missions that had a cost exceeding \$1B
there was consensus within the CMB community that a compelling CMB mission is beyond this scope.   
The aperture size and science goals we are envisioning for the CMB mission 
are most akin to EPIC-IM and CORE and we therefore believe it fits within the Probe class. 

\vspace{-0.18in}

\subsubsection{The Need for This Study} 

\vspace{-0.05in}

The EPIC-IM report from 2009 represents the US community's most recent view of the anticipated 
science reach and possible implementation of a future US space mission. There is a pressing need 
to update this view, and present this view to the next decadal panel.  

Theoretical advances and progress in physics and astrophysics gave updated 
goals for the fidelity of measurements of $E$ and $B$ modes, including measurements of inflationary 
gravitational waves, the properties of light relics, and structure formation in the universe. A 
slew of sub-orbital experiments together with the \planck\ mission have 
transformed our view of the mm-wave polarized sky, highlighting the requirement on 
thorough understanding of the foregrounds. Advances in detector technologies, multiplexed readouts,
and optical components now enable a significantly more capable mission than the one envisioned
ten years ago. And the community has vastly more experience with designs of polarimeters and 
the control of their systematic uncertainties.  A new study, based on this accumulated information and 
experience, is timely; this is the study we are proposing here. 

The US LiteBIRD team has proposed participation in LiteBIRD and recently generated its Phase A 
report. The proposal and report were conducted by 
a subset of the community for the purpose of supporting a specific mission design, within specific 
cost caps, that match JAXA plans. 

Work on our proposal, and 
the subsequent mission study, represent a collaborative effort by all interested members of the 
CMB community, including members of the LiteBIRD team. We have also reached out to our international partners 
and invited them to participate. The final report will present a consensus view of the US CMB community. 
This would be the proper input for the deliberations of the next US decadal panel. 



























