
\section{Budget Justification}
\label{sec:budget}

\subsection{General}
\label{sec:budget_general}

The majority of funding is allocated to these categories of expenses: summer salary, primarily for the PI who will 
coordinate the entire effort; travel of team members to JPL to participate in mission design sessions; 
the community workshop that will discuss the complementarity between a future space mission and sub-orbital efforts. 

\subsection{Funded Team Members}

The PI Shaul Hanany requests 4 weeks of summer salary in year 1 and 3 weeks in year 2.  We are 
also requesting 2 weeks of summer salary to support Knox, the CoI organizing the theory effort. 

Support for an administrative assistant is required due to the demand on regular staff that usually 
accompanies putting on a workshop.  This effort is above their normal departmental duties.  The assistant 
will help organize the workshop and develop a website, process registrations, secure accommodations 
and make other travel arrangements for workshop attendees, work outside their regular work hours attending 
the meeting and troubleshooting, and will have to process numerous reimbursements.  The assistant will also 
assist with the community organization tasks during the rest of the year.  

\subsection{Travel}

Travel funds are requested for the PI and four others to go to Pasadena, CA to collaborate with 
JPL�s Advanced Projects Design Team (Team X).  There will be 2 trips in each budget period.  One longer trip 
of 3 days/2nights and one shorter one of 2 days/1 night.  Both trips are based on airfare of \$500, 
lodging of \$150/night, and meals of \$64/day with 75\% on the first and last day and about \$250 for 
ground transportation and other incidentals.  This comes to \$11,000 per budget period 
(5 person-trips at \$1,200 + 5 person-trips at \$1,000).

Also included, is travel for the PI to the AAS 2018 Conference in Budget Period 2, which per 
the NRA, is where a presentation of findings will be most likely be made.  The cost of this trip is \$1,700 and is 
based on airfare to Washington, DC of \$500, \$226/night hotel  x 3, \$69/day for meals using 75\% for the first 
and last day), and then \$217 for ground transportation and incidentals.

All travel is based on the published GSA rates at the time of this proposal.

\subsection{Workshop}

A research collaboration workshop will be held in Budget Period 1, most likely in the summer of 2017.  
It will be attended by approximately 150 people, both domestic and foreign.  The cost of \$30,000 is based on 
partial travel support and local lodging for about 25 attendees at \$1,000 per person = \$25,000.  In addition, 
there will be venue rental costs and provisions for the meetings; breakfast foods, afternoon snacks and 
beverages for coffee breaks (\$5,000).  

\subsection{Publications}

We are planning to publish the results of our study. Page charges of \$1,100 are included in both budget periods.  
A likely journal is the Astrophysics Journal.  Page charges are based on publishing a a 10-page article each year.  
The current rate is \$110 for an electronic submission.  

\subsection{Communication Costs:}
This project will required several weekly telecons.  The cost budgeted is based on a monthly cost of about \$65
based on experience with other projects.


