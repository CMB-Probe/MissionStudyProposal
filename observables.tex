
\subsection{Observables}
\label{sec:observables}

\vspace{-0.05in}

\comred{{[\sc just a placeholder; copied all distortion intro here. Shaul is writing this.]}}
%
In addition to the CMB temperature and polarization anisotropy targeted by CMB imagers, {\it unique} new information about 
early-universe physics can be gained by studying the energy spectrum of the CMB~\citep{Sunyaev1970SPEC, Burigana1993, Hu1993, Chluba2011therm}. The measurements of COBE/FIRAS have shown that the average CMB spectrum is consistent with that of a blackbody to an accuracy of 4 parts in $10^{4}$ at a temperature 
$T_0=(2.726\pm 0.001)\,{\rm K}$~\citep{Mather1994, Fixsen1996}. These are the best measurements
of the average spectrum to date, but next generation CMB spectrometers are expected to greatly improve the $\mu$-distortion limits of COBE/FIRAS \citep{Kogut2011PIXIE}. Several standard processes that encode 
unique information about the Universe are expected to {\it distort} the CMB spectrum \citep[e.g.,][]{Chluba2016LCDM} at a level that is within reach of present-day technology \citep{Kogut2011PIXIE, PRISM2013WPII}. 
%
The classical distortion shapes, commonly denoted as $y$- and $\mu$-type distortions, arise from early energy release and subsequent energy exchange of CMB photons with free electrons through inverse Compton scattering \citep{Zeldovich1969, Sunyaev1970mu}. A $\mu$-distortion can only be produced in the hot and dense environment present at redshifts $z\gtrsim 5\times10^4$, while $y$-type distortions are caused at lower redshifts. This makes $\mu$-distortions a unique messenger from the early Universe. 
