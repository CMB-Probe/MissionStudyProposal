
\subsection{Baselines and Observables}
\label{sec:observables}

\vspace{-0.05in}

We are proposing to study a probe-scale mission to extract the wealth 
of cosmological information contained in the spectrum and polarization of the \ac{CMB}. 
The starting points for our study of this CMB Probe are two current-decade space missions, 
EPIC-IM and Super-PIXIE~\cite{bock2009, Kogut2011PIXIE}. EPIC-IM was presented 
to the 2010 decadal panel as a candidate \ac{CMB} imaging polarization space mission. 
It was based on a 2~m aperture telescope and 11,094 bolometric transition edge sensors. 
PIXIE is a proposed Explorer-scale mission focused on a measurement of the spectrum 
and polarization of the CMB on large angular scales. Super-PIXIE is envisioned to be a scaled up, 
more capable version of PIXIE. It consists of 4 spectrometers, each operating between 
30 and 6015~GHz with 400 15~GHz-wide bands. Improvements in technology by the next decade will enable 
the design of a mission that is more capable compared to EPIC-IM and Super-PIXIE. Therefore, all 
quantitative predictions presented in this proposal, which are based on EPIC-IM and Super-PIXIE, 
represent {\it minimum} capabilities for the CMB Probe. 

The best measurements of the \ac{CMB} spectrum -- made by COBE/FIRAS approximately 25 years ago --
show that the average CMB spectrum is consistent with that of a blackbody to an accuracy of 4 parts 
in $10^{4}$~\cite{Mather1994, Fixsen1996}. Distortions in this spectrum encode a wealth of new information.
The distortion shapes are commonly denoted as $\mu$- and $y$-types~\cite{Zeldovich1969, Sunyaev1970mu}. The 
$\mu$-distortion arises from energy release in the early Universe and can only be produced in the hot and dense 
environment present at high redshifts. This makes $\mu$-distortions a novel messenger from a redshift 
range $ z \geq 5\times10^{4} $. The $y$ distortions are caused by 
energy exchange between \ac{CMB} photons and free electrons through inverse Compton 
scattering. These originate at lower redshifts and are sensitive to the 
evolution of the large scale structure of the Universe. 

Thomson scattering at the surface of last scattering is the source of the polarization of the \ac{CMB}. It is useful 
to decompose the polarization field to two modes that are independent over the full sky, $E$ and $B$ modes. 
Together with the pattern of temperature anisotropy $T$, the \ac{CMB} thus gives three auto- and three cross-spectra. 
The {\it Planck} satellite and larger aperture ground-based instruments measured the $T$ spectrum to cosmic
variance limit for $\ell \leq ??$. Much information remains encoded in the $E$ and $B$ spectra, whose full exploration 
has just begun~\cite{planck,spt,polarbear,bicep}.   

A future \ac{CMB} Probe-scale mission will address the physics of the big bang and of quantum gravity; it will 
measure the sum of the neutrino masses, and constrain the effective number of light particle species and 
the nature of dark matter; it will probe the existence of new forms of matter at the early Universe; it will 
give new insights on the star-formation history across cosmic times, and it will provide information about 
the processes that control structure formation. In addressing these broad array of fundamental questions the 
Probe firmly fits into NASA's strategic plan as articulated by its Strategic Goal~1 ``Expand the frontiers of knowledge", 
and specifically Objective~1.6 ``Discover how the Universe works, [and] explore how it began and evolved".
%and three
%for the temperature ($T$) 
%and for the Scalar perturbations, 
%such as primordial energy density perturbations only give $E$-modes. The $B$-mode is a unique probe of early Universe 
%tensor perturbations, such as Inflation~\cite{kamionkowski97a,zaldarriaga97}. 

%\begin{table}[h]
%\small
%\centering
%%\begin{adjustbox}{width=1\textwidth,center}
%\begin{tabular}{|c|c|c|c|c|c|c|}
%\hline
%{\bf Freq} & { $\theta_{FWHM}$} & {\bf N$_{bol}^a$} & \multicolumn{2}{|c|}{\bf NET [$\pmb\mu$K $\sqrt{s}$]} & {\bf w$_p^{-1/2}$} & {\bf $\delta T_{pix}^e$ } \\ \cline{4-5}
%{\bf [GHz]} & {\bf [$'$]} & {\bf [\#]} & {\bf bolo$^b$} & {\bf band$^c$} & {\bf [$\mu$K-$'$]$^d$} & {\bf [nK]} \\ \hline
%30  & 28  & 84   & 84  & 9.2 & 14 & 83 \\ \hline
%45  & 19  & 364  & 71  & 3.7 & 5.7 & 34 \\ \hline
%70  & 12  & 1332 & 60  & 1.6 & 2.5 & 15 \\ \hline
%100 & 8.4 & 2196 & 54  & 1.1 & 1.8 & 10 \\ \hline
%150 & 5.6 & 3048 & 52  & 0.9 & 1.4 & 8  \\ \hline
%220 & 3.8 & 1296 & 59  & 1.6 & 2.5 & 15 \\ \hline
%340 & 2.5 & 744  & 100  & 3.7 & 5.6 & 33 \\ \hline
%500 & 1.7 & 1092 & 350  & 10  & 16 (140)$^f$ & 8$^g$ \\ \hline
%850 & 1.0 & 938  & 15000  & 280 & 740 (70)$^f$ & 7$^g$ \\ \hline
%{\bf Total$^h$} & & {\bf 11094} & & {\bf 0.6} & {\bf 0.9} & {\bf 5.4} \\
%\hline
%\multicolumn{3}{|l}{\footnotesize$^a$Two bolometers per focal plane pixel} & \multicolumn{4}{l|}{\footnotesize$^e$Sensitivity $\delta$T$_{CMB}$ in a $2^{\circ} \times 2^{\circ}$ pixel} \\
%\multicolumn{3}{|l}{\footnotesize$^b$Single bolometer sensitivity, CMB temperature} & \multicolumn{4}{l|}{\footnotesize$^f$Point source sensitivity in $\mu$Jy/beam (1$\sigma$) w/o confusion} \\
%\multicolumn{3}{|l}{\footnotesize$^c$Sensitivity combining all bolometers in a band} & \multicolumn{4}{l|}{\footnotesize$^g$Surface brightness sensitivity, Jy/sr, $2^{\circ} \times 2^{\circ}$ pixel (1$\sigma$) } \\
%\multicolumn{3}{|l}{\footnotesize$^d[8\pi \mbox{NET}_{bolo}^2 / (T_{mis}N_{bol})]^{1/2}(10800/\pi)$} & \multicolumn{4}{l|}{\footnotesize$^h$Combining all bands together} \\
%\hline
%\end{tabular}
%\end{adjustbox}
%\vspace{-0.13in}
%\caption{ \small \setlength{\baselineskip}{0.96\baselineskip}
%EPIC sensitivities. 
%\label{tab:epic} }
%\normalsize
%\vspace{-0.05in}
%\end{table}


%at a temperature 
%$T_0=(2.726\pm 0.001)\,{\rm K}$~\citep{Mather1994, Fixsen1996}. These are the best measurements
%of the average spectrum to date, 

%but next generation CMB spectrometers are expected to greatly improve the $\mu$-distortion limits of COBE/FIRAS \citep{Kogut2011PIXIE}. Several standard processes that encode unique information about the Universe are expected to {\it distort} the CMB spectrum \citep[e.g.,][]{Chluba2016LCDM} at a level that is within reach of present-day technology \citep{Kogut2011PIXIE, PRISM2013WPII}.  The classical distortion shapes, commonly denoted as $y$- and $\mu$-type distortions, arise from early energy release and subsequent energy exchange of CMB photons with free electrons through inverse Compton scattering \citep{Zeldovich1969, Sunyaev1970mu}. A $\mu$-distortion can only be produced in the hot and dense environment present at redshifts $z\gtrsim 5\times10^4$, while $y$-type distortions are caused at lower redshifts. This makes $\mu$-distortions a unique messenger from the early Universe. 




%In addition to the CMB temperature and polarization anisotropy targeted by CMB imagers, {\it unique} new information about 
%early-universe physics can be gained by studying the energy spectrum of the CMB~\citep{Sunyaev1970SPEC, Burigana1993, Hu1993, %Chluba2011therm}. 
