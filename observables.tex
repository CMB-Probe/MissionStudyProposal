
\subsection{Observables and Baselines}
\label{sec:observables}

\vspace{-0.05in}

We are proposing to study a probe-scale mission to extract the wealth 
of cosmological information contained in the spectrum and polarization of the \ac{CMB}. This information is unique
and can not be provided through any other astrophysical observable. 

The starting points for our study are 
two missions, EPIC-IM and Super-PIXIE~\cite{epic_im,pixie}. EPIC-IM was presented 
to the 2010 decadal panel as a candidate \ac{CMB} polarization space mission. It is based on a 2~m aperture telescope 
and 11,094 bolometric transition edge sensors. Table~\ref{tab:epic} gives the mission's frequency bands and sensitivities. 
PIXIE is a proposed Explorer scale mission to measure the spectrum and polarization of the CMB. Super-PIXIE is a scaled up, 
much more capable version of PIXIE.

The best measurements of the \ac{CMB} spectrum -- made by COBE/FIRAS approximately 25 years ago --
show that the average CMB spectrum is consistent with that of a blackbody to an accuracy of 4 parts 
in $10^{4}$~\cite{Mather1994, Fixsen1996}. Distortions in this spectrum encode a wealth of new information.
The distortion shapes are commonly denoted as $\mu$- and $y$-types. The 
$\mu$-distortion arises from energy release in the early Universe and can only be produced in the hot and dense 
environment present at redshifts $z\gtrsim 5\times10^4$. This makes $\mu$-distortions a unique messenger from a red-shift 
range that is not accessible to other probes. The $y$ distortion is caused by 
energy exchange between the \ac{CMB} photons and free electrons through inverse Compton 
scattering~\cite{Zeldovich1969, Sunyaev1970mu}. These are caused at lower redshifts and are sensitive to the 
evolution of the large scale structure. 

Thomson scattering at the surface of last scattering is the source of the polarization of the \ac{CMB}. It is useful 
to decompose the polarization field to two modes that are independent over the full sky, $E$ and $B$-modes. Scalar perturbations, 
such as primordial energy density perturbations only give $E$-modes. The $B$-mode is a unique probe of early Universe 
tensor perturbations, such as Inflation~\cite{kamionkowski97a,zaldarriaga97}. 

%at a temperature 
%$T_0=(2.726\pm 0.001)\,{\rm K}$~\citep{Mather1994, Fixsen1996}. These are the best measurements
%of the average spectrum to date, 

%but next generation CMB spectrometers are expected to greatly improve the $\mu$-distortion limits of COBE/FIRAS \citep{Kogut2011PIXIE}. Several standard processes that encode unique information about the Universe are expected to {\it distort} the CMB spectrum \citep[e.g.,][]{Chluba2016LCDM} at a level that is within reach of present-day technology \citep{Kogut2011PIXIE, PRISM2013WPII}.  The classical distortion shapes, commonly denoted as $y$- and $\mu$-type distortions, arise from early energy release and subsequent energy exchange of CMB photons with free electrons through inverse Compton scattering \citep{Zeldovich1969, Sunyaev1970mu}. A $\mu$-distortion can only be produced in the hot and dense environment present at redshifts $z\gtrsim 5\times10^4$, while $y$-type distortions are caused at lower redshifts. This makes $\mu$-distortions a unique messenger from the early Universe. 




%In addition to the CMB temperature and polarization anisotropy targeted by CMB imagers, {\it unique} new information about 
%early-universe physics can be gained by studying the energy spectrum of the CMB~\citep{Sunyaev1970SPEC, Burigana1993, Hu1993, %Chluba2011therm}. 