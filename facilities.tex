
\section{Facilities and Equipment}
\label{sec:facilities}

We are requesting to work with JPL's Team X. 

Team X systematically engineers a mission point design, to a benchmarked level of fidelity and validation, 
for a predictable cost on a predictable schedule. 

Concurrent engineering designates a method in which the minimum necessary set of relevant experts 
convenes in a shared space, operating on shared data and facilitated by an expert study lead, to conduct a 
quantitative concept study. The inputs are typically a set of performance requirements for the mission. 
The output is a reviewed study report that captures mission design; derived requirements; instrument 
selections; operations concept; component descriptions and performance for all subsystems; budgets for 
mass, power and other resources; and programmatic (schedule, cost, and risk) estimates for the mission. 
A typical study includes two dozen specialists, includes one or more client representatives, and takes a few 
weeks from initial meeting to final report, with less than ten hours of in-session, concurrent work. Because 
of its intensity, the Team X environment imposes practical limits on session duration and personnel duty cycle.

The mechanism combines people, processes, and tools in a manner refined over two decades. The line-organization 
element of the Foundry, JPL�s organization which oversees Team X, provides subsystem experts and the 
technical vetting of design models they build and maintain, so that Team X operates on a footing of technical 
consistency with how the Laboratory implements projects. The line also provides skills unique to concurrent 
engineering, including study facilitation and formulation systems engineering. The institutional element of the 
Foundry manages an IT backbone that integrates the parameters exchanged among the over a hundred design and 
analysis models; the data server, storage, archiving, and configuration management functions; method training; 
and strategic investments in continuous capability improvement. Special-purpose theaters equipped with 
networked engineering workstations and display screens are configured to support all-in conversations, 
multi-specialist sidebars, and individual yet simultaneous work. In 2009 the Foundry invested in a major renovation 
and upgrade of the Team X facilities, which now includes two fully capable study theaters.
The concurrent engineering technique, and particularly its electronically enabled setting, has been copied 
throughout the aerospace industry and beyond, and been applied to preliminary design as well as 
conceptual design stages. However, Team X remains unique in that its practitioners are JPL technical experts, 
and that since 1994 it has engineered well over a thousand concepts for Earth science, planetary, 
and astrophysics missions. Together the experts and database comprise a key resource of captured and 
tacit corporate knowledge upon which new concepts can be engineered expeditiously and confidently. 
