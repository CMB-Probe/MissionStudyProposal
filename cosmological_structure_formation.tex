
\vspace{-0.15in}
\subsubsection{Cosmological structure formation}
\vspace{-0.05in}
Understanding the evolution of cosmological structures from the formation of the
first stars to the present time is a key goal of cosmology \citep{dunlop2011}.

In particular understanding cosmological reionization, the transformation of neutral
hydrogen into an ionized state, and establishing a connection between reionization and early galaxies, will 
reveal valuable information about the star formation history and the physical processes that formed the galaxies of various luminosities and masses we see today.  Though a number of diverse observational probes have rendered a preliminary picture of the reionization epoch, details of the transition from the neutral to ionized Universe are still the subject of intense activity. 
In the current picture, early galaxies reionize hydrogen progressively throughout the entire Universe between $z \eq12$ and $z  \eq 6$,
while quasars take over to reionize helium from $z \simeq 6 to $z \simeq 2$. However many questions remain. When did the epoch of reionization
(EoR) start, and how long did it last? Are early galaxies enough to reionize the entire Universe or is another source required?
To answer these questions one can resort to using the traces left by the EoR in the cosmic microwave background (CMB) anisotropies.
More specifically, the integrated Thomson scattering optical depth, $\tau$, due to the scattering of CMB photons off free electrons after reionization, as inferred from the polarised CMB photons. The optical depth to reionization, $\tau$, places an important integral constraint on the extended reionization history. The {\it Planck}] Collaboration \citep{planck2015-XLVI,planck2015-XXXI} reported recently a value of $\tau=0.055 \pm 0.009 significantly lower than previous estimates. This suggests that an early onset of reionization is strongly disfavoured by the {\it Planck} data. The {\it Planck}] Collaboration \citep{planck2015-XXXI} showed that this result reduces the tension between CMB-based analyses and constraints from other astrophysical sources.  A cosmic variance limited measurement of E-mode polarization on large scales, possible with a probe mission, will render the most accurate determination of $\tau$ (Figure~\ref{{fig:Neff_future} shows a cosmic variance limit measurement of $\tau$ along with the current {\it Planck} limit), break the degeneracy with the neutrino mass, set stringent constraints on models of the reionization epoch, and, finally, help understanding the formation of the cosmological structures we see today.


The anisotropies in the Cosmic Infrared Background (CIB), produced by
dusty star-forming galaxies (DSFGs) in a wide redshift range, are
an excellent probe of both the history of star formation and the link between
galaxies and dark matter across cosmic time. The {\it Planck} Collaboration \citep{planck2014-XXX,planckXVIII},
derived limits on the star formation rate density that,
at redshifts z$\mathrm{\sim3}$, are about three times higher
than constraints from number counts measurements (\cite{madau2014}).
The new mission probe, by measuring CIB anisotropies with 100 times more
sensitivity than {\it Planck}, will be able to shed light on this intriguing
discrepancy, by strongly constraining the history of star formation
in the range $\mathrm{0<z<4}$, and in particular with one tenth of the {\it Planck} uncertainty
at z$\mathrm{=3}$. Moreover, it will be possible to identify and constrain a
characteristic halo mass $M_{\mathrm{eff}}$,
which determines the most efficient gas accretion and SFR, and
therefore sets the evolution of the galaxies residing within
a dark matter halo. Current models and measurements
constrain this characteristic halo mass at
$M_{\mathrm{eff}}\sim 10^{12}$ solar masses with about $\mathrm{10\%}$
uncertainty, while the new mission probe will
constrain this parameter at the percent level.\\
Moreover, because DSFGs trace the underlying dark matter
field in a broad redshift range, the CIB will
correlate with multiple dark matter
tracers such as catalogs of galaxies and quasars
\citep{serra2014,wang2015},
and diffuse maps of the $\gamma$-ray and
the X-ray background \citep{cooray2016}.
These cross-correlations will provide an additional probe of the
global star formation history, and they
will shed light on the interaction between light and matter
in a broad wavelength range.

Large-scale structure can also be probed using CMB spectral distortions measurements. In fact, the largest guaranteed distortion is caused by the associated late-time energy release of forming structures and from reionization \citep{Sunyaev1972b, Hu1994pert, Oh2003, Cen1999, Refregier2000}, imprinting a $y$-type distortion 
with $y \simeq 2\times 10^{-6}$ \citep[e.g.,][]{Refregier2000, Hill2015}. This distortion is only one order of magnitude below the current limit from COBE/FIRAS and, even with most pessimistic assumptions about foregrounds, should be clearly detected with the next-generation spectrometers we propose to study. A detection will give information about the total energy output of first stars, AGN and galaxy clusters. In particular, group-size clusters that have masses $M\simeq 10^{13}\,M_{\odot}$ contribute significantly to the signal. With temperature $k T_{\rm e}\simeq 1\,{\rm keV}$ these are still sufficiently hot to create a detectable relativistic temperature correction to the large $y$-distortion, 
which can be used to constrain the currently uncertain feedback mechanisms used in hydrodynamical simulations
of cosmic structure formation~\citep{Hill2015}. These two inevitable signals probe the low-redshift 
Universe and provide clear targets for future spectral distortions measurements and their requirements in the presence of foregrounds.

The CMB spectrum varies spatially across the sky. One source of such anisotropic distortion is related to clusters of galaxies and has already been measured by Planck~\citep{Planck2013SZ}. A combination of precise CMB imaging and spectroscopic measurements will allow observing the relativistic temperature correction of individual SZ clusters~\citep{Sazonov1998, Itoh98, Challinor98}, which will calibrate cluster scaling relations and inform our knowledge of the dynamical state of the cluster atmosphere. Finally, resonant scattering signals in the recombination \citep{Jose2005, Carlos2007Pol, Lewis2013} and post-recombination eras \citep{Kaustuv2004, Schleicher2008} can lead to spectral-spatial CMB signals that can be used to constrain the presence of metals in the dark ages and the physics of recombination. For all these applications, instrumental synergies between CMB imaging and spectroscopy need to be studied in detail. 



These studies will be key to address three of the seven key questions identified in the Astro2010 report ``New Worlds, New Horizons in Astronomy and Astrophysics'' (NAS Decadal Survey, p. 47): {\it What is the fossil record of galaxy assembly                                                                               
from the first stars to present? What are the connections between dark and luminous matter? How do cosmic structures form and evolve?}

