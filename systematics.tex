
\subsubsection{Systematic Errors}
% Brendan
\vspace{-0.05in}

Advances in detector technology since the formulation of Planck and WMAP will enable huge gains in raw 
sensitivity for a CMB probe.  To fully take advantage of this sensitivity, systematic errors must be
controlled to detect polarization signals at nano-Kelvin levels.  The proposed study will invest heavily in 
designing an instrument,  test plan, and observation strategy to address systematic errors,
gathering decades of experience of ground-, balloon-, and space-based CMB polarimetry.  The latest 
experience with \planck\ points to the following systematic error categories likely to be important for a
future space mission~\cite{PlanckintermediateXLVI}:
1. Intensity-to-polarization leakage, 2. stability, and 3. straylight. 

%Each of these is considered in light of the instantaneous signals measured by
%polarization-sensitive detectors at different times and orientations are combined to recover
%the maximum likelihood polarization signal from each point on the sky.

\textbf{Leakage} \hspace{0.1in} The CMB anisotropy signal is a factor of 1000 larger 
than the strongest possible inflationary B-mode signal (see for example Fig.~\ref{fig:clall}). 
Therefore instrumental effects that can leak
even a small fraction of an intensity fluctuation into spurious  polarization
must understood and controlled. The main effects are differences between gains of detectors
sensitive to two orthogonal polarization states, their frequency bandpass mismatch, their differential pointing
on the sky, 
and their differential antenna patterns. These differential effects need to be controlled, through 
instrument design, characterization, or data analysis to 1 part in $10^{4}$ to give a negligible contribution for 
$\sigma(r) <0.001$.  This level of control represents a factor of 100 improvement on \planck 's performance~\cite{PlanckintermediateXLVI}.


%Relative calibration requirements are likely to exceed those of
%Planck, whose High Frequency Instrument achieved of order 0.01\% (cite
%https://arxiv.org/abs/1605.02985).
%PeterM
%Bandpass mismatch is the imperfect matching of the frequency dependant gain of the two detectors or acquisition channels used to estimate the polarized signal.  Any asymmetry turns into a spurious polarization signal when viewing sources with SED different from the calibrator. For CMB-probe, this includes every foreground contaminant on the sky.  The effect can be minimized by matching the bandpasses, and residual mismatch can be estimated by measuring the bandpasses on the ground, however, for both instruments of Planck the ground based measurements of the bandpasses were found to be insufficient to determine the impact on the polarized maps. In both cases estimates of the bandpass mismatch needed to be determined as part of the mapmaking pipeline, and bandpass mismatch remains one of the most problematic systematics in the pipelines. CMB-probe will need a more robust program to mitigate this problem on ground, simulate the impact, and to deal with it in the flight data.

A second order complication of unmatched bandpasses is unmatched far sidelobe contamination, leading to a spurious polarized component from bright unpolarized signals far from the main lobe.

%Bandpass mismatch between
%polarized detectors gives an error in relative calibration when the
%SED of the sky signal differs from that of the calibrator.  This gives
%a spatial dependence to the leakage, potentially complicating the
%component separation.

These systematics are likely to drive the instrumental
requirements on the optical system as well as the uniformity of the
bandpass of each polarimeter.  Calibration requirements will also be
set by limiting these systematics: particularly on the
knowledge of the polarization parameters (such as cross-polar leakage
and the angle of polarization sensitivity), as well as measurement of
the beam shape (in general a function of the SED of the observed source).
These systematic effects can potentially be mitigated by modulation of the sky signal in
such a way that allows complete reconstruction of the 
polarized sky signal using each photometer, for example, using a
half-wave plate.  

\textbf{Stability.}  Given the need to avoid light from the Sun, Earth, and Moon, the full reconstruction 
of the polarized sky will necessarily involve combination of measurements made at times
separated by months, requiring stability of the response of the instrument on corresponding time scales.  
This systematic error puts requirements on control of thermal drifts of spacecraft temperatures, to
mitigate thermal emissivity changes and thermoelastic deformation of telescope structures.  
The cryogenic operating temperatures of detectors or reference calibration loads must be controlled
adequately as well. Careful design of the scan strategy can shorten the time
scales needed for stringent stability, for example Planck's scan strategy traced out great circles which overlapped on
1~minute timescales, giving a shorter effective time scale for stability requirements. 

Additionally, the space radiation environment is modulated by the solar activity and
can introduce drifts in the cryogenic thermal environment as well as
introducing correlated transients in detectors and readout
electronics.  The design of the instrument must account this
environment, which following Planck is much better understood.

\textbf{Straylight.}   The brightest cm-wave and mm-wave sources in
the sky (such as the Sun, Moon, planets, and Galactic center) passing
into the far sidelobes of the telescope (defined as the response of a
detector from a source more than a few degrees from the optical axis)
in a sky-synchronous way can create a spurious polarization signal.
%PeterM
The far sidelobe response can be reduced by the optical design and baffling, but will always be present at a non-trivial level.  The Planck experience is instructive here as well.  The detailed GRASP model of the telescope, convolved with the sky model, predicted sidelobe contamination at a visible (10's of microK) level in the 30 GHz maps, which was observed in difference maps.  As a result the LFI maps have had an estimate of the sidelobe contamination removed from the timelines as part of the mapmaking process. The more stringent requirements for CMB-probe will necessitate at least this level of mitigation.  A major limitation to the analysis of far sidelobe contamination in the Planck data has been the lack of bright enough on-orbit sources to validate the GRASP simlation as adjusted by optical and bandpass parameters estimated on-orbit. Finding a way to better validate the FSL model on-orbit for CMB-probe may be critical to successfully removing FSL contamination.

%The far sidelobes can be reduced by optical design and baffling, but diffraction ensures that this
%off-axis response will always be present at some level.  Measurement on
%the ground can allow for some correction, and design of the scan
%strategy can modulate the sidelobe pickup in a different way from the
%true on-axis polarization signal, allowing its removal.

