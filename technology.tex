
\subsection{State of Technologies}
\label{sec:technologies}

\vspace{-0.05in}

The imager version of the probe consists of the following main technical elements: a telescope with an effective aperture size 
of $\lesssim1.5$~m, a focal plane consisting of thousands of detectors, coolers that provide a focal plane temperature between 0.1 and 0.3~K, 
and a multiplexed readout system with which a handful of wires are used to readout hundreds or thousands of detectors. Additional 
elements could include filters and potentially lenses and polarization modulators. 
The spectrometer version is also a cryogenic mission, and has two main elements: the spectrometer, and the cold load that provides its
absolute calibration. Both versions have the standard complement of spacecraft bus features to provide pointing control 
and sensing, telemetry, and power. 

Relative to \planck, which was the last CMB imaging cryogenic mission, the most significant advances have been made in 
developing detector and readout technologies, and in optical components. While \planck\ had 52 bolometric polarization 
sensitive detectors, sub-orbital experiments now routinely implement thousands. Aided by large throughput optical 
systems, and large lenses Stage3 experiments will implement
tens of thousands. By the time the Probe flies a focal plane with tens of thousands 
of detectors, in which few electrical lines readout thousands of detectors will be standard, well-tested technology. 
A key technical question for the implementation of the probe is whether the need to reject systematic uncertainties 
will require the implementation of an active polarization modulator. 

{\bf Arrays of Detectors} \hspace{0.1in} 
Most modern sub-orbital experiments use TES bolometers, with thousands of detectors with TRL>5.  
These have been successfully implemented with a variety of optical coupling schemes such as horns, 
contacting lenslets, and antenna arrays. Some instruments have deployed newer technology with arrays of muti-chroic pixels;  
several frequency bands are detected through the same focal plane pixel. Sub-orbital instruments have achieved map-depths on 
small patches of sky that are within a factor of 3 of the depth that the baseline mission targeted across the entire sky. A new detector 
technology using kinetic inductance inductors (KIDs) is emerging, which may have benefits in simplicity of fabrication and a 
scalability to arrays with hundreds of thousands of elements.  
A future CMB probe with these technologies would be immensely more powerful than that envisioned with the baseline mission. 

{\bf Readout }  \hspace{0.1in} Currently the CMB field uses two families of readout technologies: frequency domain multiplexing (fMUX) 
and time domain multiplexing (TDM).  Both offer $~64$ channels per readout module and have mature TRLs having been 
flown on sub-orbital missions.    Emerging technologies include code-division multiplexing (based on TDM) which could 
substantially reduce the power requirements for the readout electronics; and microwave fMUX which promises to 
incorporate $>1000$ channels of TES detectors or KIDs per multiplexed module, greatly simplifying focal plane integration 
and reducing the cryogenic load on the cold stage.  Microwave readout places higher demands on the warm readout 
electronics than lower frequency alternatives. Lower power field programmable gate array- and graphical processor unit-based 
systems are under active development.

{\bf Polarization Modulators and Other Optical Components}  \hspace{0.1in} A polarization modulator presents 
an attractive means to reject a host of systematic uncertainties. Some sub-orbital experiments have used modulators 
and experience with their operation, efficacy, and the systematic errors they present, is growing. For use with the Probe,
the modulators will need to have high polarization efficiency over a broad bandwidth. Approximately 100\% fractional 
bandwidth has been demonstrated. Optical systems that incorporate refractive elements can realize higher throughput 
than reflectors alone; the use of refractors -- or a modulator -- requires broad-band anti-reflection coatings. Groups 
have developed specialized sprays and techniques to fabricate sub-wavelength structures. Most of these 
technologies have TRL$\geq$5.  

{\bf Spectrometer} \hspace{0.1in} 
The spectrometer is comprised of a number of polarizing Fourier transform spectrometers, each with an absolute reference calibrator.
The spectrometer builds on the COBE/FIRAS mission using mature technology with TRL $\geq$6.
Multi-moded optics, concentrators, detectors, and calibrators have been demonstrated. The detector readout is copied from the 
that used for the Hitomi mission. 

{\bf Cryogenics }  \hspace{0.1in} For providing an operating temperature of 0.1~K: an open cycle dilution 
refrigerator, a European technology, was flown on \planck . 
A closed cycle version is under development (also in Europe) and has TRL 3-4. A Goddard continuous adiabatic 
demagnetization refrigerator (ADR) will soon be flown on a balloon payload.  The Hitomi spacecraft operated a 
staged version of this a ADR. For higher operating temperatures, refrigerator technologies are standard, but 
suitability for mission loads will be assessed during the study. 
%The TRL is above 6 for refrigerators for all other tempera

%  We will study the requirements and tradeoffs of these mature cooling technologies in a mission plan.



% Section 1.7.1 Study Plan (McMahan, O�Brient)

%Participants in this group have partial support through outside sub-orbital projects to develop technologies relevant to the probe.  We will thus be well positioned to provide current assessments of TRL as the study progresses.  With this insight, we will provide technical support and assistance to the team designing the mission and instruments.  At the same time, we will identify technologies that are in need of development on a 10 year time scale and seek out ways to facilitate that development with our sub-orbital projects.  In this way, this team will efficiently leverage existing projects to advance the case for a flight mission.


%We will study the instrument configuration and evaluate systematics, sensitivity, risks and requirements for this configuration.

%It also introduces the added risk of a rotating 
%element in space  while introducing risk 
%Filters, lenses, and polarization modulators are rapidly maturing 
%technologies that have the potential to improve sensitivity, reduce focal plane mass, manage thermal loads, and mitigate 
%important systematic effects.  Antireflection (AR) coatings on silicon lenses with up to 3:1 ratio bandwidth have been deployed, 
%similar lenses will soon be flown on balloons, and coatings on larger alumina lenses will soon be fielded on the ground.   
%These AR coating approaches and metal-mesh metamaterials have led to new realizations of broad-band half-wave plate 
%(HWP) modulators which complement other approaches including variable phase polarization modulators.  
%Continuous and stepped HWPs have been flown on sub-orbital missions to mitigate systematics.  Deployed filters 
%include metal-mesh, AR coated crystals, plastic layers and hybrids of these approahes.    We will study mission 
%configurations that include these technologies and optimize the system to minimize mass, cooling requirements, 
%maximize sensitivity, and identify methods for control of systematic effects with and with out a polarization modulator.
