
\subsection{State of Technologies}
\label{sec:technologies}

\vspace{-0.05in}

The imager version of the probe consists of the following main technical elements: a telescope with an effective aperture size 
of $\lesssim1.5$~m, a focal plane consisting of thousands of detectors, coolers that provide a focal plane temperature between 0.1 and 0.3~K, 
and a multiplexed readout system with which a handful of wires are used to readout hundreds or thousands of detectors. Additional 
elements could include filters and potentially lenses and polarization modulators. 
The spectrometer version is also a cryogenic mission, and has two main elements: a spectrometer, and the cold load 
that provides its absolute calibration. Both versions have the standard complement of spacecraft bus features 
to provide pointing control and sensing, telemetry, and power. 

\planck, which was the last CMB imaging cryogenic mission, had 65 polarization sensitive detectors. 
The most significant advances since \planck~have been in developing detector and readout technologies, and optical components.
The baseline imager, enjoying technologies of a decade ago, had $\sim$30 times the sensitivity of \planck.  
%Sub-orbital experiments now routinely implement thousands. 
As the paragraphs below describe, a mission with today's technologies would already be 
more powerful than the baseline mission.  The CMB Probe promises to be orders of magnitude more 
sensitive than \planck.  \\
{\bf Arrays of Detectors} \hspace{0.1in} 
Most modern sub-orbital experiments use TES bolometers, with thousands of detectors with TRL$\geq 5$. 
HEMT amplifiers, which are a competitive technology below 100~GHz, also has high TRL.   
The bolometric arrays have been successfully implemented with a variety of optical coupling schemes such as horns, 
contacting lenslets, and antenna arrays. Some instruments have deployed newer technology with arrays of `muti-chroic pixels'. 
With this technology several frequency bands are detected through the same focal plane pixel. As of now, arrays with up 
to 3 bands and 6 detectors per pixel are being used. 
%Sub-orbital instruments have achieved map-depths on 
%small patches of sky that are within a factor of 3 of the depth that the baseline mission targeted across the entire sky. 
A new detector 
technology using kinetic inductance inductors (KIDs) is emerging, which may have benefits in simplicity of fabrication and 
scalability to arrays with hundreds of thousands of elements.  \\
{\bf Readout }  \hspace{0.1in} Two families of readout technologies are in use: frequency- and time-domain multiplexing, 
fMUX, and TDM, respectively.  Both offer $~64$ channels per readout module and have mature TRLs having been 
flown on sub-orbital missions.   Emerging technologies include code-division multiplexing (based on TDM), which could 
substantially reduce the power requirements for the readout electronics; and microwave fMUX which promises to 
incorporate $>1000$ channels of TES detectors or KIDs per multiplexed module. The microwave fMUX simplifies 
focal plane integration and reduces the cryogenic load on the cold stage but places higher 
demands on the warm readout electronics compared to lower frequency alternatives. 
Lower power systems with field programmable gate arrays, graphical processor units, and application 
specific integrated circuits are under active development.  \\
{\bf Polarization Modulators and Other Optical Components}  \hspace{0.1in} A polarization modulator presents 
an attractive means to reject a host of systematic uncertainties. Some sub-orbital experiments have used modulators 
and experience with their operation, efficacy, and the systematic errors they present, is growing. For use with the Probe,
the modulators will need to have high polarization efficiency over a broad bandwidth.  A fractional bandwidth of $\sim$100\%  
has been demonstrated. Optical systems that incorporate refractive elements can realize higher throughput 
than reflectors alone; the use of refractors -- or a modulator -- requires broad-band anti-reflection coatings. Groups 
have developed specialized sprays and techniques to fabricate sub-wavelength structures. Most of these 
technologies have TRL$\geq$5.  \\
{\bf Spectrometer} \hspace{0.1in} 
The polarizing Fourier transform spectrometer builds on the COBE/FIRAS mission using mature technology with TRL $\geq$6. 
The baseline spectrometer we have assumed here 
is comprised of a number of individual spectrometers, each with its own absolute reference calibrator, 
Multi-moded optics, concentrators, detectors, and calibrators have been demonstrated. The detector readout is copied from the 
that used for the Hitomi mission. But the Probe version may combine multiple spectrometer beams within a single telescope. 
How to achieve that will be part of this study. \\
{\bf Cryogenics }  \hspace{0.1in} For providing an operating temperature of 0.1~K: an open cycle dilution 
refrigerator, a European technology, was flown on \planck . 
A closed cycle version is under development (also in Europe) and has TRL 3-4. A Goddard continuous adiabatic 
demagnetization refrigerator (ADR) will soon be flown on a balloon payload.  The Hitomi spacecraft operated a 
staged version of this a ADR. For higher operating temperatures, refrigerator technologies are standard, but 
suitability for the mission thermal loads will be assessed during the study. 



