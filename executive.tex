
\subsection{Executive Summary}
\label{sec:executive}

\vspace{-0.05in}

We are proposing to study a probe-scale mission to extract the wealth 
of physical, cosmological, and astrophysical information contained in the spectrum and polarization of the \ac{CMB}. 
The CMB Probe will search for the signature of primordial gravitational waves from the big bang 
and thus probe quantum gravity. It will constrain the effective 
number of light particle species, with precision only available to CMB measurements. 
With its full sky coverage, it will measure the sum of the neutrino masses, doubling 
the significance of a $2\sigma$ detection reachable 
by experiments that measure only smaller portions of the sky. It will probe 
the nature of dark matter and the existence of new forms of matter at the early universe.  It will 
give new insights on the star-formation history across cosmic times. And it will provide information about 
the processes that control structure formation on all scales, from clusters
of galaxies to the collapse of a protostellar core.    
% add something about galactic science and magnetic fields
With high sensitivity, access to the entire sky, broad frequency coverage,  
and exquisite control of systematic effects the Probe is best poised to realize the fidelity of measurements 
necessary to extract these science goals. 

The last US CMB community's consensus assessment of the case for 
and design of a space mission took place 8 years ago. %years from the time this study will conclude. 
Since then theoretical considerations, available data from \planck\ and sub-orbital measurements, technology advances, 
and plans for new sub-orbital experiments have changed the landscape considerably. 
We propose to provide the 2020 decadal panel with a fresh expert assessment. 

The scope of science we envision for the Probe is achievable within the approximate technical 
envelope of our 2010 baseline mission, which was near \$900M. This scope of science 
is also targeted by a recently submitted proposal for a European-based mission that has 
similar cost.  Both of these missions have broader science reach than a 
more focused Japanese-led mission, which 
is near the \$400M limit. We thus assess that the CMB mission is in the Probe cost window. 

The mission study is led by Steering 
and Executive Committees made up of scientists who built COBE, WMAP, \planck , and the 
leading sub-orbital experiments in the world; by scientists who processed, analyzed, and simulated data from 
these experiments; and who interpreted the results and put them in a physics and cosmology context. 
It is open to all member of the CMB community, and will represent hundreds of person years of 
accumulated knowledge, expertise, and experience. 

