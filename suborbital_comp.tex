%%\ref{sec:science}

Ground based experiments are already approaching the level of
sensitivity of the CMB Probe over limited portions of the sky ($<1$\%
of the full sky), and a limited range in frequency.
%cite bicep/keck, spt
Balloon-borne experiments and ground based telescopes at
mid-latitudes are now beginning to extend these measurements to
significant fractions of the full sky. 
%cite ACT
Currently funded balloon borne
experiments will add to this sensitive data at frequencies above 200
GHz which will help characterize Galactic dust, while ground based
experiments in Chile will add critical information at lower
frequencies to constrain synchrotron emission.
%cite Spider Fraisse et al, Piper, Ebex, CLASS, ACT
The Stage-4 experiments
will extend these measurements to much larger areas, and to angular
scales smaller than $5^\prime$, with benefits to the CMB Probe as
described in Section \ref{sec:science}.

The CMB Probe will add to these complete sky coverage, fidelity to the
largest angular scales, comprehensive frequency coverage and exquisite
control of systematic effects.  As evidenced by the experience of
Bicep and {\it Planck}, the joint analyses are becoming increasingly
important to fully exploit the scientific value of CMB data sets.  As
part of this mission study, we will organize a workshop to organize
this effort, and investigate the ways in which the ground and
sub-orbital balloon programs can best complement the CMB probe.

%describe the state of measurements - Planck, Bicep/Keck/CLASS, SPT/ACT, Spider, and the future with S4/Simons Observatory mix of large and small aperture, future balloons Spider/Piper/Ebex 

