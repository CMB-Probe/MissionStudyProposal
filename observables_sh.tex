
\subsection{Observables and Baselines}
\label{sec:observables}

\vspace{-0.05in}

Thomson scattering at the surface of last scattering is the source of the polarization of the \ac{CMB}. It is useful 
to decompose the polarization field to two modes that are independent over the full sky, $E$ and $B$ modes. 
Together with the pattern of temperature anisotropy $T$, the \ac{CMB} thus gives three auto- and three cross-spectra. 
The {\it Planck} satellite and larger aperture ground-based instruments measured the $T$ spectrum to cosmic
variance limit for $\ell \leq 1500$. Much information remains encoded in the $E$ and $B$ spectra, whose full exploration 
has just begun~\cite{planck2015parameters, Hanson2013,PB_BB,bicep2Bmode, bkp2015}.  
%$B$-mode signals that arise from the 
%conversion of $E$ polarization to $B$ due to gravitational lensing of photons as they traverse the Universe, 
%the so called 'lensing $B$ mode', has been detected within the last few years.    

The best measurement of the \ac{CMB} spectrum -- made by COBE/FIRAS approximately 25 years ago --
shows the average CMB spectrum is consistent with that of a blackbody to an accuracy of 5 parts 
in $10^{5}$~\cite{Mather1994, Fixsen1996}. Distortions in this spectrum encode a wealth of new information.
The distortion shapes are commonly denoted as $\mu$- and $y$-types~\cite{Zeldovich1969, Sunyaev1970mu}. The 
$\mu$-distortion arises from energy release in the early universe and can only be produced in the hot and dense 
environment present at high redshifts. This makes $\mu$-distortions a novel messenger from a redshift 
range $ z \geq 5\times10^{4} $. The $y$ distortions are caused by 
energy exchange between \ac{CMB} photons and free electrons through inverse Compton 
scattering. These originate at lower redshifts and are sensitive to the 
evolution of the large scale structure of the universe. 

Quantitative predictions in this proposal are based on two current-decade space missions, 
EPIC-IM and Super-PIXIE~\cite{bock2009, Kogut2011PIXIE}. EPIC-IM was presented 
to the 2010 decadal panel as a candidate \ac{CMB} imaging polarization mission. 
It was based on a 1.4~m effective aperture telescope and 11,094 bolometric transition edge sensors. 
PIXIE is a proposed Explorer-scale mission focused on a measurement of the spectrum 
and polarization of the CMB on large angular scales. Super-PIXIE is envisioned to be a scaled up, 
more capable version of PIXIE. It consists of 4 spectrometers, each operating between 
30 and 6000~GHz with 400 $\sim$15~GHz-wide bands. Improvements in technology by the next decade will enable 
the design of a mission that is more capable compared to EPIC-IM and Super-PIXIE. Therefore, all 
quantitative predictions presented in this proposal, which are based on EPIC-IM and Super-PIXIE, 
represent {\it minimum} capabilities for the CMB Probe. 



%A future \ac{CMB} Probe-scale mission will address the physics of the big bang and of quantum gravity; it will measure the sum of the neutrino masses, and constrain the effective number of light particle species and the nature of dark matter; it will probe the existence of new forms of matter at the early universe; it will give new insights on the star-formation history across cosmic times, and it will provide information about the processes that control structure formation. In addressing these broad array of fundamental questions the Probe firmly fits into NASA's strategic plan as articulated by its Strategic Goal~1 ``Expand the frontiers of knowledge", and specifically Objective~1.6 ``Discover how the universe works, [and] explore how it began and evolved".
