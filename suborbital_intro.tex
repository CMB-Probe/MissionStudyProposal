%
% Complementarity between the Probe and Sub-orbital efforts
%

The remarkable scientific yield resulting from past CMB observations
has motivated significant investments in current and future
experiments which are designed to realize the full potential of this
unique probe of fundamental physics.  These ground-based and
balloon-borne experiments are designed to exploit the comparative
advantages of the sub-orbital platforms over an orbital mission, while
providing the design heritage and experience necessary to maximize the
probability of success of an orbital mission.

For the ground-based efforts, these include combinations of {\it i)}
provision for large primary apertures and therefore high angular
resolution, {\it ii)} extremely long observation times, {\it iii)}
flexibility to rapidly deploy new technologies, and {\it iv)}
allowance for detector formats that are relatively unconstrained by
mass and power limitations.  In aggregate, these enable extremely low
noise measurements of small scale anisotropies with high redundancy
over large areas of the sky.

The balloon-borne missions {\it i)} extend the frequency reach of the
ground based telescopes, {\it ii)} enable high fidelity measurements
on larger angular scales than can be probed from the ground, and {\it
  iii)} grant inexpensive access to a space-like environment,
providing heritage for future space missions as well as experience in
dealing with the analysis of data that are representative of a space
mission.

In this way, the sub-orbital programs complement and multiply the
scientific return of the proposed orbital mission, while reinforcing
its technical preparedness.  A robust program of sub-orbital
experimentation has proven a vital component in the success of all
three generations of previous CMB orbital missions -- COBE, WMAP and
{\it Planck}. Building on this heritage, the current (Stage-3) and
planned (Stage-4) sub-orbital experiments are well poised to play a
similar role for the CMB Probe, which will provide definitive
measurements of the full sky from the largest angular scales to the
$5^\prime$ scale of the beam.





%Current and future sub-orbital missions

% strengths of sub-orbital
% high ell
% tau?
% technology demonstration

